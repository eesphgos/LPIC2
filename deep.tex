\documentclass{article}
\usepackage{listings}
\usepackage{xcolor}
\usepackage{graphicx}
\usepackage{fancyhdr}
\usepackage{geometry}
\usepackage[utf8]{inputenc}
\usepackage[english]{babel}

\geometry{a4paper, margin=1in}
\pagestyle{fancy}
\fancyhf{}
\rhead{CW GIT - Network Administration}
\lhead{\today}
\cfoot{\thepage}

\title{Comprehensive Network Administration Guide}
\author{Ehsan Esmaeili}
\date{November 18, 2024}

\begin{document}

\maketitle
\tableofcontents
\newpage

\section{IP Address Assignment}

\subsection{Types of Address Assignment}
\begin{itemize}
    \item \textbf{Manual} $\rightarrow$ Static assignment
    \item \textbf{Dynamic} $\rightarrow$ DHCP server
\end{itemize}

\subsection{Network Configuration Commands}
\begin{lstlisting}[language=bash, backgroundcolor=\color{gray!10}, basicstyle=\ttfamily\small]
# Display all network interfaces
ifconfig -a
# Show routing table
route -n
ip route show

# Network configuration files
# Ubuntu/Debian:
/etc/network/interfaces
# CentOS/RHEL:
/etc/sysconfig/network-scripts/

# Restart networking service
systemctl restart networking
\end{lstlisting}

\subsection{Secure Shell (SSH)}
\begin{itemize}
    \item Remote management protocol (TCP/22)
    \item Encrypted communication
\end{itemize}

\begin{lstlisting}[language=bash, backgroundcolor=\color{gray!10}, basicstyle=\ttfamily\small]
# Install SSH server
apt-get install openssh-server  # Debian/Ubuntu
yum install openssh-server     # CentOS/RHEL

# Check listening ports
netstat -ntulp
n -> numeric    t -> TCP
u -> UDP        l -> listening
p -> PID

# DNS configuration
cat /etc/resolv.conf

# Generate UUID for network interfaces
uuidgen
\end{lstlisting}

\section{DHCP Server Configuration}

\subsection{ISC DHCP Server Setup}
\begin{lstlisting}[language=bash, backgroundcolor=\color{gray!10}, basicstyle=\ttfamily\small]
# Configure DHCP server interface
vi /etc/default/isc-dhcp-server
INTERFACES="interface_name"

# DHCP configuration file
vi /etc/dhcp/dhcpd.conf
# Set IP range, DNS, reservations

# Manage DHCP service
systemctl restart isc-dhcp-server
systemctl status isc-dhcp-server

# Check DHCP port
netstat -ntulp | grep :67

# Remove packages
apt-get remove package_name
apt-get purge package_name

# DHCP logs
cat /var/log/syslog | grep -i dhcp
cat /var/lib/dhcp/dhcpd.leases
\end{lstlisting}

\section{Name Resolution}

\subsection{Types of Name Servers}
\begin{itemize}
    \item \textbf{DNS Server (BIND)}: FQDN/DNS name $\rightarrow$ IP
    \item \textbf{WINS Server}: NetBIOS $\rightarrow$ IP
\end{itemize}

\subsection{Local Name Resolution}
\begin{lstlisting}[language=bash, backgroundcolor=\color{gray!10}, basicstyle=\ttfamily\small]
# Local hosts file
vi /etc/hosts

# Name service switch configuration
vi /etc/nsswitch.conf
\end{lstlisting}

\section{BIND DNS Server}

\subsection{DNS Concepts}
\begin{itemize}
    \item \textbf{FQDN Structure}: www.example.com.
    \item \textbf{Local DNS Server}: Forwarding/Recursive
    \item \textbf{DDNS}: Dynamic DNS
    \item \textbf{Master/Slave DNS}: Zone transfer
\end{itemize}

\subsection{DNS Query Tools}
\begin{lstlisting}[language=bash, backgroundcolor=\color{gray!10}, basicstyle=\ttfamily\small]
nslookup example.com
host example.com
dig example.com
dig @DNS_SERVER_IP example.com
\end{lstlisting}

\subsection{Resource Records}
\begin{itemize}
    \item \textbf{SOA}: Start of Authority (Zone transfer)
    \item \textbf{NS}: Name Server (Authoritative)
    \item \textbf{A}: IPv4 address record
    \item \textbf{AAAA}: IPv6 address record
    \item \textbf{PTR}: Reverse lookup
    \item \textbf{MX}: Mail exchange
    \item \textbf{CNAME}: Alias/Canonical name
    \item \textbf{TXT/SPF}: Text records
\end{itemize}

\subsection{BIND Installation and Configuration}
\begin{lstlisting}[language=bash, backgroundcolor=\color{gray!10}, basicstyle=\ttfamily\small]
# Install BIND
apt-get install bind9 bind9utils      # Debian/Ubuntu
yum install bind bind-utils          # CentOS/RHEL

# Check installed packages
rpm -qa | grep bind                  # CentOS/RHEL

# Main configuration files
vi /etc/bind/named.conf.options
vi /etc/bind/named.conf.local

# Enable recursion
recursion yes;

# Reload configuration
rndc reload
systemctl restart bind9
systemctl status bind9
\end{lstlisting}

\subsection{Master DNS Server Setup}
\begin{lstlisting}[language=bash, backgroundcolor=\color{gray!10}, basicstyle=\ttfamily\small]
# Zone configuration
zone "example.com" {
    type master;
    file "/etc/bind/db.example.com";
};

# Create zone database file
vi /etc/bind/db.example.com

$TTL 3600
@ IN SOA example.com. root.example.com. (
    10      ; Serial
    1200    ; Refresh
    300     ; Retry
    86400   ; Expire
    3600 )  ; Minimum TTL

@       IN NS      ns1.example.com.
ns1     IN A       192.168.56.10
www     IN A       192.168.56.60
ftp     IN A       192.168.56.70
\end{lstlisting}

\subsection{Reverse DNS Zone}
\begin{lstlisting}[language=bash, backgroundcolor=\color{gray!10}, basicstyle=\ttfamily\small]
# Reverse zone configuration
zone "56.168.192.in-addr.arpa" {
    type master;
    file "/etc/bind/db.56.168.192";
};

# PTR records
10 IN PTR ns1.example.com.
60 IN PTR www.example.com.
70 IN PTR ftp.example.com.

# Test reverse lookup
dig @localhost -x 192.168.56.60
\end{lstlisting}

\subsection{Time Zone Configuration}
\begin{lstlisting}[language=bash, backgroundcolor=\color{gray!10}, basicstyle=\ttfamily\small]
# Debian/Ubuntu
vi /etc/timezone
Asia/Tehran

cp /usr/share/zoneinfo/Iran /etc/localtime

# CentOS/RHEL
timedatectl list-timezones
timedatectl set-timezone Asia/Tehran
timedatectl
\end{lstlisting}

\subsection{Slave DNS Server Configuration (CentOS)}
\begin{lstlisting}[language=bash, backgroundcolor=\color{gray!10}, basicstyle=\ttfamily\small]
vi /etc/named.conf

# Listen on all interfaces
listen-on port 53 { any; };

# Allow queries from any
allow-query { any; };

# Slave zone configuration
zone "example.com" {
    type slave;
    masters { 192.168.56.10; };
    file "slaves/db.example.com";
};

# Enable named service on boot
systemctl enable named
systemctl disable named
\end{lstlisting}

\subsection{DNSSEC with TSIG}
\begin{lstlisting}[language=bash, backgroundcolor=\color{gray!10}, basicstyle=\ttfamily\small]
# Generate TSIG key
cd /etc/bind
dnssec-keygen -a HMAC-MD5 -b 128 -n HOST transferkey

# Create key file
vi /etc/bind/named.conf.tsig
key "transferkey" {
    algorithm HMAC-MD5;
    secret "KEY_STRING";
};

# Include in main config
include "/etc/bind/named.conf.tsig";

# Configure zone transfer
allow-transfer { key "transferkey"; };
\end{lstlisting}

\section{Apache Web Server}

\subsection{Virtual Hosting Methods}
\begin{enumerate}
    \item Different IP addresses
    \item Same IP, different ports
    \item Same IP and port, different hostnames (Name-based)
\end{enumerate}

\subsection{Apache Installation}
\begin{lstlisting}[language=bash, backgroundcolor=\color{gray!10}, basicstyle=\ttfamily\small]
# Debian/Ubuntu
apt-get install apache2

# CentOS/RHEL
yum install httpd

# Configuration directories
/etc/apache2/              # Debian/Ubuntu
/etc/httpd/               # CentOS/RHEL
\end{lstlisting}

\subsection{Virtual Host Configuration}
\begin{lstlisting}[language=bash, backgroundcolor=\color{gray!10}, basicstyle=\ttfamily\small]
# Debian/Ubuntu structure
/etc/apache2/sites-available/
/etc/apache2/sites-enabled/

# Enable site
a2ensite example.conf
systemctl reload apache2

# Virtual host example
<VirtualHost 192.168.56.20:80>
    ServerName www.example.com
    DocumentRoot /var/www/example
</VirtualHost>
\end{lstlisting}

\subsection{Authentication and Redirection}
\begin{lstlisting}[language=bash, backgroundcolor=\color{gray!10}, basicstyle=\ttfamily\small]
# Create password file
htpasswd -c /etc/apache2/.htpasswd username

# Directory protection
<Directory "/var/www/secure">
    AuthType Basic
    AuthName "Restricted Area"
    AuthUserFile /etc/apache2/.htpasswd
    Require valid-user
</Directory>

# URL redirection
Redirect /old /new
\end{lstlisting}

\subsection{SSL/TLS Configuration}
\begin{lstlisting}[language=bash, backgroundcolor=\color{gray!10}, basicstyle=\ttfamily\small]
# Generate self-signed certificate
openssl req -x509 -nodes -days 365 -newkey rsa:2048 \
    -keyout /etc/httpd/private.key \
    -out /etc/httpd/public.crt

# Apache SSL configuration
SSLCertificateFile /etc/httpd/public.crt
SSLCertificateKeyFile /etc/httpd/private.key
\end{lstlisting}

\section{SSH Configuration}

\subsection{Passwordless SSH Authentication}
\begin{lstlisting}[language=bash, backgroundcolor=\color{gray!10}, basicstyle=\ttfamily\small]
# Generate SSH key pair
ssh-keygen -t rsa

# Copy public key to remote server
ssh-copy-id user@remote_host

# Test connection
ssh user@remote_host
\end{lstlisting}

\section{Routing}

\subsection{Route Types}
\begin{itemize}
    \item \textbf{Network Route}: 192.168.56.0/24
    \item \textbf{Host Route}: 192.168.56.20/32
    \item \textbf{Default Route}: 0.0.0.0/0
\end{itemize}

\subsection{Enable IP Forwarding}
\begin{lstlisting}[language=bash, backgroundcolor=\color{gray!10}, basicstyle=\ttfamily\small]
# Temporary enable
echo 1 > /proc/sys/net/ipv4/ip_forward

# Permanent enable
vi /etc/sysctl.conf
net.ipv4.ip_forward = 1

# Apply changes
sysctl -p
\end{lstlisting}

\section{Firewall Configuration}

\subsection{UFW (Ubuntu)}
\begin{lstlisting}[language=bash, backgroundcolor=\color{gray!10}, basicstyle=\ttfamily\small]
ufw status
ufw allow 22/tcp
ufw deny 23/tcp
ufw delete RULE_NUMBER
ufw reset
\end{lstlisting}

\subsection{FirewallD (CentOS)}
\begin{lstlisting}[language=bash, backgroundcolor=\color{gray!10}, basicstyle=\ttfamily\small]
systemctl start firewalld
firewall-cmd --list-all
firewall-cmd --add-service=ssh --permanent
firewall-cmd --add-port=80/tcp --permanent
firewall-cmd --reload
\end{lstlisting}

\section{iptables}

\subsection{iptables Tables and Chains}
\begin{itemize}
    \item \textbf{Tables}: filter, nat, mangle, raw, security
    \item \textbf{Built-in Chains}: INPUT, OUTPUT, FORWARD, PREROUTING, POSTROUTING
\end{itemize}

\subsection{Common iptables Commands}
\begin{lstlisting}[language=bash, backgroundcolor=\color{gray!10}, basicstyle=\ttfamily\small]
# List rules
iptables -L -n -v
iptables -t nat -L

# Flush rules
iptables -F
iptables -t nat -F

# Save rules
iptables-save > /etc/iptables/rules.v4

# Restore rules
iptables-restore < /etc/iptables/rules.v4
\end{lstlisting}

\subsection{Example Rules}
\begin{lstlisting}[language=bash, backgroundcolor=\color{gray!10}, basicstyle=\ttfamily\small]
# Block ICMP from specific IP
iptables -A INPUT -p icmp -s 192.168.56.20 -j REJECT

# Allow SSH from specific network
iptables -A INPUT -p tcp --dport 22 -s 192.168.56.0/24 -j ACCEPT

# Default policies
iptables -P INPUT DROP
iptables -P FORWARD DROP
iptables -P OUTPUT ACCEPT
\end{lstlisting}

\subsection{NAT Configuration}
\begin{lstlisting}[language=bash, backgroundcolor=\color{gray!10}, basicstyle=\ttfamily\small]
# Masquerading (Internet sharing)
iptables -t nat -A POSTROUTING -o eth0 -j MASQUERADE
iptables -A FORWARD -i eth1 -j ACCEPT

# Port forwarding
iptables -t nat -A PREROUTING -p tcp --dport 2222 \
    -j DNAT --to-destination 192.168.56.30:22
\end{lstlisting}

\section{DHCP Relay Agent}
\begin{lstlisting}[language=bash, backgroundcolor=\color{gray!10}, basicstyle=\ttfamily\small]
# DHCP server configuration
subnet 172.20.1.0 netmask 255.255.255.0 {
    range 172.20.1.111 172.20.1.120;
    option routers 172.20.1.2;
}

# DHCP relay configuration
apt-get install isc-dhcp-relay  # Ubuntu
yum install dhcp                # CentOS

# Configure relay
vi /etc/default/isc-dhcp-relay
SERVERS="192.168.56.10"
INTERFACES="eth0 eth1"
\end{lstlisting}

\section{LVM (Logical Volume Manager)}

\subsection{LVM Basic Operations}
\begin{lstlisting}[language=bash, backgroundcolor=\color{gray!10}, basicstyle=\ttfamily\small]
# Create physical volumes
pvcreate /dev/sdb1 /dev/sdc1

# Create volume group
vgcreate vg_data /dev/sdb1 /dev/sdc1

# Create logical volume
lvcreate -n lv_www -L 10G vg_data

# Create filesystem
mkfs.ext4 /dev/vg_data/lv_www

# Mount filesystem
mount /dev/vg_data/lv_www /var/www

# Extend logical volume
lvextend -L +5G /dev/vg_data/lv_www
resize2fs /dev/vg_data/lv_www
\end{lstlisting}

\section{RAID Configuration}

\subsection{RAID Levels}
\begin{itemize}
    \item \textbf{RAID 0}: Striping (performance)
    \item \textbf{RAID 1}: Mirroring (redundancy)
    \item \textbf{RAID 5}: Distributed parity
    \item \textbf{RAID 10}: Mirroring + striping
\end{itemize}

\subsection{Software RAID with mdadm}
\begin{lstlisting}[language=bash, backgroundcolor=\color{gray!10}, basicstyle=\ttfamily\small]
# Create RAID 1 array
mdadm --create --verbose /dev/md0 \
    --level=1 --raid-devices=2 /dev/sdb1 /dev/sdc1

# Create filesystem
mkfs.ext4 /dev/md0

# Monitor RAID status
cat /proc/mdstat
mdadm --detail /dev/md0

# Save configuration
mdadm --detail --scan > /etc/mdadm.conf
\end{lstlisting}

\section{Samba File Sharing}

\subsection{Samba Server Configuration}
\begin{lstlisting}[language=bash, backgroundcolor=\color{gray!10}, basicstyle=\ttfamily\small]
# Install Samba
apt-get install samba smbclient  # Ubuntu
yum install samba samba-client   # CentOS

# Configure Samba
vi /etc/samba/smb.conf

[shared]
    path = /srv/samba/shared
    valid users = @smbgroup
    read only = no
    create mask = 0775

# Add Samba user
smbpasswd -a username

# Restart service
systemctl restart smbd nmbd
\end{lstlisting}

\section{Kernel Modules}

\subsection{Module Management}
\begin{lstlisting}[language=bash, backgroundcolor=\color{gray!10}, basicstyle=\ttfamily\small]
# List loaded modules
lsmod

# Module information
modinfo module_name

# Load module
modprobe module_name
insmod /path/to/module.ko

# Remove module
rmmod module_name
modprobe -r module_name

# Update module dependencies
depmod -a
\end{lstlisting}

\section{Nginx Web Server}

\subsection{Nginx Installation and Configuration}
\begin{lstlisting}[language=bash, backgroundcolor=\color{gray!10}, basicstyle=\ttfamily\small]
# Install Nginx
apt-get install nginx  # Ubuntu
yum install nginx     # CentOS

# Configuration structure
/etc/nginx/


# Test configuration
nginx -t

# Reload configuration
nginx -s reload
systemctl reload nginx
\end{lstlisting}

\subsection{Nginx as Reverse Proxy}
\begin{lstlisting}[language=bash, backgroundcolor=\color{gray!10}, basicstyle=\ttfamily\small]
upstream backend {
    server backend1.example.com:8080;
    server backend2.example.com:8080;
    server backend3.example.com:8080;
}

server {
    listen 80;
    location / {
        proxy_pass http://backend;
    }
}
\end{lstlisting}

\section{IPv6 Configuration}

\subsection{IPv6 Address Types}
\begin{itemize}
    \item \textbf{Global Unicast}: 2001:db8::/32
    \item \textbf{Link-local}: fe80::/64
    \item \textbf{Unique Local}: fd00::/8
    \item \textbf{Multicast}: ff00::/8
\end{itemize}

\subsection{IPv6 Network Configuration}
\begin{lstlisting}[language=bash, backgroundcolor=\color{gray!10}, basicstyle=\ttfamily\small]
# Static IPv6 configuration
iface eth0 inet6 static
    address 2001:db8::10
    netmask 64

# IPv6 ping
ping6 2001:db8::1

# IPv6 DNS configuration
host example.com AAAA
dig example.com AAAA
\end{lstlisting}

\section{FTP Server (vsftpd)}

\subsection{vsftpd Configuration}
\begin{lstlisting}[language=bash, backgroundcolor=\color{gray!10}, basicstyle=\ttfamily\small]
# Install vsftpd
apt-get install vsftpd  # Ubuntu
yum install vsftpd      # CentOS

# Main configuration
vi /etc/vsftpd.conf

# Enable write access
write_enable=YES

# User access control
userlist_enable=YES
userlist_file=/etc/vsftpd.user_list
userlist_deny=NO

# Restart service
systemctl restart vsftpd
\end{lstlisting}

\section{Fail2ban Intrusion Prevention}

\subsection{Fail2ban Configuration}
\begin{lstlisting}[language=bash, backgroundcolor=\color{gray!10}, basicstyle=\ttfamily\small]
# Install Fail2ban
apt-get install fail2ban

# Configuration files
/etc/fail2ban/jail.conf
/etc/fail2ban/jail.local

# Example SSH protection
[sshd]
enabled = true
port = ssh
filter = sshd
logpath = /var/log/auth.log
maxretry = 3
bantime = 600

# Manage service
systemctl restart fail2ban
fail2ban-client status
\end{lstlisting}

\section{OpenVPN}

\subsection{OpenVPN Configuration}
\begin{lstlisting}[language=bash, backgroundcolor=\color{gray!10}, basicstyle=\ttfamily\small]
# Install OpenVPN
apt-get install openvpn  # Ubuntu
yum install openvpn      # CentOS

# Generate static key
openvpn --genkey --secret static.key

# Server configuration
dev tun
ifconfig 10.8.0.1 10.8.0.2
secret static.key

# Client configuration
remote server.example.com
dev tun
ifconfig 10.8.0.2 10.8.0.1
secret static.key
\end{lstlisting}

\section{NFS (Network File System)}

\subsection{NFS Server Configuration}
\begin{lstlisting}[language=bash, backgroundcolor=\color{gray!10}, basicstyle=\ttfamily\small]
# Install NFS server
apt-get install nfs-kernel-server  # Ubuntu
yum install nfs-utils              # CentOS

# Export directory
vi /etc/exports
/shared  *(rw,sync,no_subtree_check)

# Apply exports
exportfs -ra

# Start services
systemctl restart nfs-server
\end{lstlisting}

\section{Squid Proxy Server}

\subsection{Squid Configuration}
\begin{lstlisting}[language=bash, backgroundcolor=\color{gray!10}, basicstyle=\ttfamily\small]
# Install Squid
apt-get install squid

# Basic configuration
vi /etc/squid/squid.conf

# Allow all clients
http_access allow all

# Cache configuration
cache_dir ufs /var/spool/squid 100 16 256

# Access control list
acl localnet src 192.168.0.0/16
http_access allow localnet

# Restart service
systemctl restart squid
\end{lstlisting}

\section{Mail Server (Postfix, Dovecot)}

\subsection{Postfix Configuration}
\begin{lstlisting}[language=bash, backgroundcolor=\color{gray!10}, basicstyle=\ttfamily\small]
# Install Postfix
apt-get install postfix mailutils

# Main configuration
vi /etc/postfix/main.cf

# Test mail
echo "Test message" | mail -s "Test" user@example.com

# Mail aliases
vi /etc/aliases
admin: user1, user2
newaliases
\end{lstlisting}

\subsection{Dovecot IMAP/POP3}
\begin{lstlisting}[language=bash, backgroundcolor=\color{gray!10}, basicstyle=\ttfamily\small]
# Install Dovecot
apt-get install dovecot-imapd dovecot-pop3d

# Configuration
vi /etc/dovecot/conf.d/10-mail.conf
mail_location = maildir:~/Maildir

# SSL configuration
vi /etc/dovecot/conf.d/10-ssl.conf
ssl = required
ssl_cert = </etc/ssl/certs/dovecot.pem
ssl_key = </etc/ssl/private/dovecot.pem

# Restart service
systemctl restart dovecot
\end{lstlisting}

\section{Monitoring Tools}

\subsection{Network Monitoring}
\begin{lstlisting}[language=bash, backgroundcolor=\color{gray!10}, basicstyle=\ttfamily\small]
# Port scanning
nmap localhost
nc -vz hostname port

# Bandwidth monitoring
iftop
nload

# Network testing
iperf -s  # Server
iperf -c server_ip  # Client
\end{lstlisting}

\end{document}