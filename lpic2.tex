\documentclass{article}


\title{CW GIT}

\date{2024 nov 18}

\author{Ehsan Esmaeili}

\newpage
\begin{document}

\section*{section one: ip assignment}

\subsection*{type of addr assignment}

manual -> static

dynamic -> DHCP server

ifconfig -a show you all interface

route -n show table of route

ip route show also show table

/etc/network/interfaces to see seting of interface in ubonto

/etc/syscongih/network-scripts/ for centos

systemctl restart networking

secure shell (ssh) => remote managment tcp/22 encrypt

install openssh-server

apt-ger install openssh-server

netstat -ntulp

n -> numeric

t-> tcp

u -> udp

l -> listening

p -> PID

cat /etc/resolv.conf  shot you dns server

uuidgen -> this command make a uuid to set it in to ifcfg-enp0s in centos

\section*{DHCP}

vi /etc/default/isc-dhcp-server set name of interface for lissten to dhcp

INTERFACES="name of interface"

/etc/dhcp/dhcpd.conf you can set range for ip addressing or dns or set reservation an ip

systemctl restart isc-dhcp-server

systemctl status isc-dhcp-server

netstat -ntulp | grep :67

apt-get remove name
apt-get purge name

cat /var/log/syslog | grep -i dhcpdiscover -offer- request - ach 

cat /var/lib/dhcp/dhcpd.leases

systemctl restart sshd

\section*{Name resolution}

Name ==> ip

www.google.com ---> 82.153.6.8

Name sever:

1.DNS server (BIND) FQDN or DNS name --- > ip

2.WINS server  NetBIOS ---> ip

vi /etc/hosts add an ip and name infront of it  to set name resolution

vi /etc/nsswitch change priority of dns server or hosts

\section*{Bind}

www.min.ir.""

local DNS sever (forwarding DNS server) ---> Other DNS server (Forwarder DNS server)
 this work named Recursion

DDNS

DNS master and slave with BIND

Master DNS server ==> Name Registration + Name resolution

Slave DNS server ==> Name resolution



www -> host name --> can be 63 char first 15 char for NetBIOS

.mi.ir -> DNS Suffix --> can be 255 char

nslookup sitename to get ip

host sitename to get ip

dig sitename to get ip

dig @ip to ask ip

Resource Record:

SOA (main): =====> Zone transfer 
            Serial number:
            Refresh interval: 1200 s
            Retry interval: 300 s
            Expire: 1 Day to s
            Cache TTL: 3600 s


NS (main) ===> name server ====>autoritative 

A ====> FQDN ---->IPv4

AAAA  ====> FQDN ---->IPv6

PTR ====> IP ===> FQDN

SRV

MX

CName(Alias)

TXT

SPF

dig @4.2.2.4 AAAA www.google.com

apt-get install bind9 bind9utils

rpm -qa | grep bind 

vi /etc/bind/named.conf.options

add "recursion yes ;" at ent of {}

rndc reload ---> if succesful mean every config file ok

systemctl status bind9

systemctl restart bind9 or rndc reload

\subsection*{set server as master}

vi /etc/bind/named.conf.local

to those file add 4 next lines

zone "anisa.co.ir" {
        type master;
        file "/etc/bind/db.anisa.co.ir";% ==> database
};

to crate a database for dns server

vi/etc/bind/db.anisa.co.ir % new db file

to those file add 11 next lines

\$TTL   3600
@       IN      SOA     anisa.co.ir.        root.anisa.co.ir.  (
                        10       ;serial%after; is command
                        1200     ;refresh
                        300      ;retry
                        86400    ;expire
                        3600 )    ;cash ttl   
@       IN      NS      ubuntu.anisa.co.ir.
ubuntu.anisa.co.ir.     IN      A       192.168.56.10
www.anisa.co.it.        IN      A       192.168.56.60
ftp.anisa.co.it.        IN      A       192.168.56.70

systemctl restart bind9 %restart dns server after each change

rndc reload %to see every thing is ok

if rndc is none succesful use systemctl status

or use
journalctl -xe

dig @localhost www.anisa.co.ir %ask as localhost to test

dig @localhost ftp.anisa.co.ir %ask as localhost to test

dig @localhost pop3.anisa.co.ir %ask as localhost to test this is most be faile

if wriht 192.168.56.10(dns server) in localhos place then ask dns server

\subsection*{set dns server by reverse ip addressing}

192.168.56.0/24 ==> 56.168.192.in-addr.arpa

then set PTR to convert id to fqdn

vi /.. /named.conf.local

add 

zone "56.168.192.in-addr.arpa" {
            type master ;
            file "/etc/bind/db.56.168.192";
            
};

cp db.anisa.co.ir ./db.56.168.192

vi db.56.168.192

delete 4 last line (with that have ip)

them add 

10      IN      PTR     ubuntu.anisa.co.ir.
60      IN      PTR     www.anisa.co.ir
70      IN      PTR     ftp.anisa.co.ir
80      IN      PTR     pop3.anisa.co.ir

systemctl restart bind9

rndc reload

dig @localhost -x 192.168.56.60 %-x revers

%===> set time zone and timing
vi /etx/timezone ==> Asia/Tehran

cd /use/share/zoneinfo

cp ./Iran /etc/localtime

date

%=====>in centos
timedatectl list-timezones

timedatectl set-timezone Asia/Tokyo

timedatectl

to set master and slave dns server the both timezone mostbe same

\subsection*{make slave dns server (centos)}

vi /etc/named.conf

add%

in line 13 centos listen in 127.0.0.1:53 that mean centos listen to itself
them delete 127.0.0.1 and write any to listen to another

in lin 21 allow-query most be to any instead of localhosr 
them whrit in "any;"

after default zoon than define at first add%

zone "anisa.co.ir" {
        type slave;
        masters { 192.168.56.10; };
        file "db.anisa.co.ir";
};

in ubuntu%

vi /etc/bind/db.anisa.co.ir

chande serial 2 unit%becaus be wanrt to make 2 change 

add a ns record after first ns record

@       IN      NS      centos.anisa.co.ir
centos.anisa.co.ir.     In      A       192.168.56.20
systemctl restart bind9%for ubuntu

systemctl restart named% named is bind but in centos

netstat -ntulp | grep :53 | grep named %to see changed setting


cat /var/log/messages | grep -i transfer %you can see log transfer in centos

dig @localhost pop3.anisa.co.ir

dig @192.168.56.10 pop3.anisa.co.ir

dig @192.168.56.20 pop3.anisa.co.ir

systemctl enable named % auto start named sevice in next reboot
systemctl disable named

\subsection*{DNSsec}

%in master 
TSIG (Transfer SIGniture) = a secure chanel between master and slave

cd /etc/bind

dnssec-keygen -a HMAC-MD5 -b 128 -n HOST -r /dev/urandom transerkey

                |           |      |        |

cat Ktransferkey.+157+02007.private | grep Key > ./named.conf.tsif

vi named.conf

add at end

include "/etc/bind/named.conf.tsig";

vi named.conf.local

in anisa zone

add afrer file""

allow-transfer {key "transferkey";};

systemctl rstart bind9

rndc reload

we maby got an error from named.conf.tsig file

change Key: 231313131.. to

key "transferkey" {
        algorithm HMAC-MD5;
        secret "231313131..";
};

systemctl restart bind9

rndc reload

%in slave

systemctl status sshd

ss -ntulp | grep :22

%in master

scp ./named.conf.tsig root@192.168.56.20:/root/

%in slave

cat named.conf.tsig >> /etc/named.conf

vi /etc/named.conf

4dd move those line to above of options

and beetween options and key add

server 192.168.56.10{
        keys { transferkey; };
};


systemctl restart named

%in master 

systemctl restart bind9

rndc reload

%in slave

systemctl restart named

\section*{Apache}

socket base application

Protocol:IP:Port

http://www.google.com:80

if we want to make two site

bui.com
mai.ir

1) diffrent IP add ===> bui.com --> 192.168.56.10 , mai --> 192.168.56.20

2) same IP address and diffrent port ===> bui.com >> 192.168.56.10:8080 , mai >> 192.168.56.10:8090

3) same IP & Prot ====> diffrent host name (header) ==> 192.168.56.10:8080 --> bui.com , mai.ir

package name in red hat = httpd and in debian = apache2

and config file with same name in the /etc

<virtualHost 192.168.56.20:20>
        DocumentRoot /var/www/html
</virtualHost 192.168.56.20:20>

in /etc/apache2/ports.cofn ---> Listen to any thing

you can also see html file in /var/www

in www html folder is default and you can add another folder to more page

in /etc/apache2/site-avalible you can add config file

to show change in site you most linked mysite in site-enable to config file than in dite-avalible

you also can make link by this command "a2endite "name of config file""

systemctl reload apache2

systemctl restart apache2

to open site with name of sit (not ip)

first in /apache2/site-enabled/ change mysite file and add a line in every server

ServerName www.mai.ir

them go to /bind/maned.conf.local

add new one for bui and mui

so go to make db of these zone
and change it

:%s/word/newwork/g

above line to change same work to new word 
in vi per each cope

systemctl restart bind9

rndc reload

systemctl  restart apache2

them go to centos (client)

vi /etc/sysconfig/network-scripts/ifcfg-enp0s3

and add DNS

DNS1=192.168.56.10

systemctl restart network

authentication methods :

1)username password ===> /etc/passwd, ldap server, kerveros, NIS, Active Directory

2)smart card ===> EAP

3)Certificate ===> CA

AAA server : ====> RADIUS server

Authenticate Authorized Accounting

authenticate for apache2


make a (name)secret file in /var/www/mai/

echo "<h1> welcome to secret page</h1>" > ./secret/index.html

htpasswd -c /etc/apache2/amirpass amir

vi /etc/apache2/site-enabled/mysite

in a virtualhost zone add these line

<Directory "/var/www/maserati/secret" >
        AuthType Basic
        AuthName "PLZ enter username and password"
        AuthUserFile /etc/apache2/amirpass
        Require valid-user 
</Directory>

systemctl restart apache2

make a (name)policy in /var/www/mai/

echo "<h1>welcom to policy page </h1>" > ./policy/index.html

vi /etc/apache2/sites-enabled/mysite.conf

in a virtual host zone add

Redirect /secret /policy


systemctl restart apache2


ssl setup in centos

yum install httpd mod_ssl

cd /etc/httpd

cd conf

vi httpd.conf

line 42 listen in 80 port

line 119 Ducumentroot "/var/www/html"

cd /var/www/html

echo "<h1>welcom to centos web server </h1>" > ./index.html

systemctl restart httpd

if error journalctl -xe

vi /etc/httpd/conf.d/ssl.conf

to set site in https

ca /etc/httpd

openssl req -x509 -nodes -days 365 -newkey rsa:2048 -keyout ./private.key -out ./public.crt

vi /etc/httpd/conf.d/ssl.conf

line 100 changto ==> SSLCertificateFile /etc/httpd/puvlic.crt

line 107 chang to ==> SSLCertificateKeyFile /etc/httpd/private.key

systemctl restart httpd

netstat -ntulp | grep :443


\section{Secur SHell (ssh)}

remote managment:

1) Telnet ===> 23 ---> Plaintext

2) SSH ===> 22 ---> Encrypt >> user,pass   password-less(ssh key)

3) RDP ===> 3389

4) SNMP ===> 

dpkg -l | grep openssh-server

apt-get install openssh-server

cd /etc/ssh

vi sshd_config --> permitlogin >> yes

systemctl restart sshd

netstat -ntulp | grep :22

%in centos (user)

ssh root@192.168.56.10

ssh root@192.168.56.10 -p 2254 (on port 2254)

exit

vi /etc/selinux/config

in line 7 =disabled

\subsection*{password less in ssh}

in ~/.ssh

ssh-keygen

ssh-copy-id -i id_rsa.pub root@192.168.56.10

ssh root@192.168.56.10


\section*{routing}

type of route

1)network route 192.168.56.0/24

2)host route 192.168.56.20/32

3)default route 0.0.0.0/0.0.0.0

priority ==> H - N - D

priotity ==> static - dynamic

static router -- dynamic routing protocol ----> dynamc router

adverisement     RIP,EIGRP,OSPF,.. BGP,IS-IS

Announcement

to see os is router or not

cat /proc/sys/net/ipv4/ip_forward if == 0 = disable

echo 1 > /proc/sys/net/ipv4/ip_forwardd

to permanent routing

vi /etc/sysctl.conf

ling 28 del #

\section*{IPTable , firewall}


\subsection*{firewall}
ufw status

ufw allow 22/tcp

to see rules ufw status

and ports in /etc/servises

ufw status numbered

ufw delete 4

ufw reset

but in centos

systemctl start firewalld

firewall-cmd --list-ports

firewall-cmd --services

firewall-cmd --remove-service=ssh

firewall-cmd --permanent --remove-service=ssh

firewall-cmd --reload

firewall-cmd --permanent --add-port=22/tcm

firewall-cmd --reload

\subsection*{IPTable}

iptavles:(IPchanges) and we have priority

1)Filter===> Block, Accept Packet

2)NAT =====> Internet sharing , Port mapping (Forwarding) , Readirect

3)Mangle

4)Raw

5)Security


every table have some Chain

Chains:
                        _
1)INPUT                 |

2)FORWARD               |

3)OUTPUT                |==> Built-in

4)PREROUTING            |

5)POSTROUTINg           |
                        _


man iptables

iptables -t filter -n -L

iptables -t filter -F

iptables -t filter -X

iptables -t filter -nL

inpables --table nat -nL

Define Rules :

-p protocol ---> tcp, udp, icmp, all

-s source address

-d destination address

-i Inbound-Interface

-o Outbound-Interface

-j (jump)Target ---> Accept, Reject(error message) , Drop (no error msg), Redirect, SNAT, DNAT, Masquerade

-A Append

-I Insert

-t choose table (default filter)

-D delete

apt-get install iptables-persistent for permanent in ruleing

rule writing :

iptables -t filter -A INPUT -p icmp -s 192.168.56.20 -j REJECT

iptables -t filter -nL

iptables -t filter -A INPUT -p tcp --destination-port 22 -j REJECT

iptables -t filter -F

iptables -t filter -A INPUT -p tcp --destination-port 22 -s 192.168.56.30 -j REJECT

iptables -t filter -I INPUT 5(default 1) -p icmp -s 192.168.56.20 -j DROP

iptables -t filter -nL --line-numbers

iptables -t filter -D INPUT 3

iptables -t filter -F

iptables -t filter -A OUTPUT -p icmp -d 192.168.56.30 -j DROP

ss -ntulp | grep :22

change port num in /etc/ssh/sshd_config

systemctl restart sshd

iptables -t nat -A PREROUTING -i enp0s3 -p tcp --dport 22 -j REDIRECT --to-port 2370(or else)

iptables -t filter -P INPUT DROP %change default

iptables -t filter -A INPUT -p tcp --destination-port 22 -s 192.168.56.1 -j ACCEPT


to save all of them

iptables-save 

iptables-save > /etc/iptables/rules.v4

\subsection*{NAT}

int nat os we most write two rule one in nat table and on in filter tables


iptables --table nat --append POSTROUNTING --out-interface enp0s8 --jump MASQUERADE

iptables --table filter --append FORWARD --in-interface enp0s3 --jump ACCEPT


iptables -t nat -A POSTROUTING -o enp0s8 -j MASQUERADE

iptables -t filter -A FORWARD -i nep0s3 -j ACCEPT


\subsection*{Port Mapping}


iptables -t nat -A PREROUTING -i enp0s8 -p tcp --destination-port 2222 -j DNAT --to-destination 192.168.56.30:2370

iptables -t nat -A PREROUTING -i enp0s8 -p tcp --destination-port 3333 -j DNAT --to-destination 192.168.56.20:2480

iptables --t filter -A FORWARD -i enp0s8 -p tcp --destination-port 2370 -j ACCEPT

iptables --t filter -A FORWARD -i enp0s8 -p tcp --destination-port 2480 -j ACCEPT

\section*{DHCP relay ageng/host (router)}

at first in dhcp sesrver os in /etc/dhcp/dhcpd.conf

add these line after subnet and host

subnet 172.20.1.0 netmask 255.255.255.0 {
        range 172.20.1.111 172.20.1.120;
        option routers 172.20.1.2;
}

in router os

apt/yum install dhcp

route add -net 172.20.1.0/24 gw 192.168.56.20 dev enp0s3

enable routing in /proc/....

cp /lib/systemd/system/dhcrelay.service /etc/systemd/system/

vi /etc/systemd/system/dhcrelay.service

in line 9 add to end

192.168.56.10

systemctl --system daemon-reload

systemctl restart dhcrelay

cat /var/lib/dhcp/dhcpd.leases %to see log


\section*{LVM (logical volume manager)}

ls /dev/sd*

lsblk

fdisk /dev/sdb \% set the lvm tag - 8e

fdisk /dev/sdc

fdisk /dev/sdd

pvs \% physical volum

vgs \% volume group

lvs \% logical volume

pvdisplay

vgdisplay

lvdisplay

pvcreate /dev/std1 /dev/sdc1 /dev/sdd1

vgcreate bigdatachunk /dev/sdb1 /dev/sdc1

pvs

vgs

lvcreate --name logical1 --size 6G bigdatachunk

lvs

mkfs.ext4 /dev/bigdatachunk/logical1

mkdir logical1

mount /dev/bigdatachunk/logical1 ./logical1

df -h

cd logical1

echo "LVM test" > ./lvm.txt

vi /etc/fstab to permanent mounting

tune2fs -l /dev/bigdatachunk/logical1 to get some information like UUID

cd ..

lvextend -L +1G /dev/bigdatachunk/logical1

df -h %you can see filesystem not extend

resize2fs /dev/bigdatachunk/lgical1

df -h

--> to decreas lv size without any problem at first 
you most unmount lv them check lv decrase stat with e2fsck
if not give any error you can decreas size
before unmount you most go out of derectory

umount /dev/bigdatachunk/logical1

e2fsck -ff /dev/bigdatachunk/logical1

fsck \% to solve problem

echo \$? to see status of last

resize2fs /dev/bigdatachunk/logical1 4G

lvreduce -L -3G /dev/bigdatachunk/logical1

lvs

df -h

mount /dev/bigdatachunk/logical1 ./logical1

df -h

vgextend bigdatachunk /dev/sdd1

pvs

vgs

df -h

lvremove /dev/bigdatachunk/logical1

umount /dev/bigdatachunk/logical1

lvremove /dev/bigdatachunk/logical1

lvs

vgs

vgreduce bigdatachunk /dev/sdd1

vgs

vgreduce bigdatachunk /sdc1

vgs

vgreduce bigdatachunk/sdb1

vgremove bigdatachunk

vgs

pvs

pvremove /dev/sdb1 /dev/sdc1 /dev/sdd1

ls /dev/sd*

\section*{RADE (Redundant Array of Independent Disks)}

RADE0 (Srtiping)

RADE1 (Mirroring)

RADE-4: Dedicated Parity 

RADE-5: Distributed Parity

RADE 10 - 1 , 0

RAID0 (2-32 disk)--> performance in read and write 
but no file tlorance 
(a data devide by 2 and half in 
disk 0 and half in disk 1)

RADE1 (2 disk)--> no performance in write 
but have performance in read but have backup 
(data in disk 1 same as in disk 0 (mirror))
in RADE1 have 50 percent over head --> 120 G --> 60G in RADE1

RADE 4 (3-32 disk)--> if we have n disk 1/n use for save parity 
and wehave 100/n percent overhead and a single data save
distribut between another disk

RADE 5 like RADE 4 but in 4 a disk save all parity and
in 5 each disk save a part of parity

in 4 , 5 if we lost two or more disk cant recaver it

RADE 10 --> RADE 1 + 0

in terminal

sha

fdisk /dev/sdb --> p --> t --> fd (RADE)-->w

fdisk /dev/sdc --> p --> t --> fd (RADE)-->w

fdisk /dev/sdd --> p --> t --> fd (RADE)-->w

cat /proc/mdstat (file of RADE)

mdadm --create --verbose /dev/md0 --level=1 --rade-devices=2 /dev/sdb1 /dev/sdc1

fdisk /dev/md0

mkfs.ext4 /dev/md0

mkdir raid1

tune2fs -l /dev/md0 | grep UUID

tune2fs -l /dev/md0 | grep UUID >> /etc/fstab

vi /etc/fstab

change lastlin to UUID=gsgsdsfg\dots    /root/raid1      ext     defaults    0    0

mdadm --detail --scan

mdadm --detail --scan > /etc/mdadm.conf

init 6

cat /proc/mdstat

df -h

cd raid1

echo "raid 1 sample test " > ./raid1.txt

mdadm --detail /dev/md0 

if a disk knocked out in RADE 1

mdadm --detail /dev/md0

cat /proc/mdstat

mdadm --manage /dev/md0 --add /dev/sdd1

if a disk born in RADE 5
first creat raid 5

mdadm --creat --verbose /dev/md --leve=5 --raid-devices=3 /dev/sdb2 /dev/sdc2 /dev/sdd2

fdisk /de/md1

mkfd.ext4 /dev/md1

mkdir raid5

tune2fs -l /dev/md1 | grep UUID >> /etc/fstab

in /etc/fstab 
change last line to 
UUID=hffhs...       /root/raid5/    ext4    defaults        0       0

mdadm --detail --scan > /etc/mdadm.conf

init 6

cat /proc/mdstat

df -h

mdadm --detail /dev/md1

cd raid 5

echo "raid 5 sample text" > ./raid5.tex

now time to lose disk

mdadm --detail /dev/md1
%add disk
lsblk

fdisk /dev/sdb

mdadm --manage /dev/md1 --add /dev/sdb2

mdadm --detail /dev/md1

\section*{Samba}

file server --> a os that share file
common internet file system (CIFS) = samba

two daemon = nmbd udp 137/138, smbd=tcp/139 or tcp/445

nmblookup

samba package for server

smbclient package,samba-client for client first for ubuntu and secont for centos

apt-get install samba smbclient

make a folder in windos in c drive with name (winfileserver)

then share folder with properties in share tab

and set permision is security tab

if we use \$ at the end of name to share 
thats hiden share

we can see all share with \texttt{\\}localhost 
command in run 

also can see hiden share with \texttt{\\}localhost\ name of share
compmgmt.msc see all share

in compmgmt.msc in tab local user and group

in user folder add new user and give adminstraitor permissing

in linux

smbclient -L 192.168.56.20 -U richard(name of user that add)

smbclient //192.168.56.20/winfileserver -U richard

if you get some error like (nt status login failed)

vi /etc/samba/smb.conf

in line 30 add

client min protocol = SMB2
client max protocol = SMB3
(if no error not write these line)

systemctl restart smbd (debian)

smbclient //192.168.56.20/winfileserver -U richard

smb: > help

smb: > dir

smb: > get winserver.txt

smb: > exit

ls

echo "sample teslfsjfj l" > ./linuxfile.txt

smbclient //192.168.56.20/winfileserver -U richard

smb: > put linuxfile.txt

smb: > dir

this part is finishet lets change role of linux and windoes

mkdir /linuxfileserver

chmod 777 /linuxfileserver

echo "slshflflsl" > ./linuxfileserver/linuxsercer.txt

useradd -d /home/maziyar -m -s /bin/bash maziyar

passwd maziyar

vi /etx/samba/smb.conf

at end of file add

[linuxfileserver]
        path = /linuxfileserver
        valid users = maziyar
        read only = no
        create mask = 0777
        directory madk = 0777

smbpasswd -a maziyar

systemctl restart smbd

in windows

in run

\texttt{\\}192.168.56.10

testparm to see status of samba

\section*{Manipulating modules(in centos)}

Driver = Module > .ko

lsmod (list of modules)

cdrom                   42556    1    sr_mod

mean sr_mod depended to cdrom

cd /lib/modules

uname -r

cd /s10.0fs  (answer of uname command)

cd net

cd ipv4

lsmod | grep cdrom

modinfo sr_mod

modinfo cdrom

remove module == rmmod or modprobe -r

inser module == 

insmod ==> 1-full path

or 

modprob ==> 1-module name, 2-Dependency

/lib/modules/kernel-version/modules.dep =>1) ko path, 2)Dependency

rmmod cdrom

modprob -r cdrom

rmmod sr-mod

modprobe -r cdrom

lsmod

/lib/modules/3.10.0-353......./modules.dep

modinfo sr_mod (take address)

insmod (write address)

modinfo cdrom (take address)

insmod (addr)

insmod (addr sr-mod)

rmmod sr-mod

rmmod cdrom

modprobe sr-mod

lsmod | grep sr-mod

depmod -a (update db of mdules)

\section*{ kernel compilation (in centos)}

linux-A.B.C.tar.xz

A major release

B minor release

C patch level

in centos to install kernel

yum groupinstall "Develpment Tools"
yum install ncurses-devel qt-devel

in ubuntu

apt-get install build-essential

cd /usr/src

tar -xvjf linux-3.6.4.tar.bz2    (gzip .gz -z, bzip2 .bz2 -j, xz .xz -J)

creating a .config file

make config (not recommended)

make menuconfig

make xconfig and gconfig

make oldconfig (not recommended)

cd linux-3.6.4 (extracted file)

use one of make config at up

old config make backup and them make new config

make mrproper (delete all config file)

after make config file enter

make zImage/bzImage (b = big for new kernel)

make modules

make modules_install

make install


\section*{Nginx}

in ubuntu

%delete apache file

apt-get remove apache2

apt-get purge apache 2

maby ger some warning we most del these warning

cd /etc rm -rf apache2

apt-get install nginx

cd /etc/nginx/sites-available

rm -f default

cd /var/www

rm -rf html/

%make needed file

mkdir anisa

cd anisa

mkdir site1 site2

echo "<h1>welcomgljlsg</h1>" > ./site1/index.html

echo "<h1>welcomgljlsg</h1>" > ./site2/index.html

cd /etc/nginx/sites-available

vi mysites.conf

add lines

server {
        listen 192.168.56.10:80;
        location / {
                root /var/www/anisa/site1;
        }
}
server {
        listen 192.168.56.100;
        locarion / {
                root /var/www/anisa/site2;
        }
}

get out

cd ../sites-enabled

ln -s ../sites-available/mysites.conf .

nginx -t (show your mistak)

systemctl restart nginx

%two site with same addr and port

vi mysites.conf 

set both locatino as 192.168.56.10:80

and theses line for both after listen line

server_name www.site1.com;

server_name www.site2.com;

nginx -t

cd /etc/bind

vi name.cconf.local

add new zone

zone "site1.com" {
        type master;
        file "/etc/bind/db.site1.com";
};

zone "site1.com" {
        type master;
        file "/etc/bind/db/sites2.com";
};

get out

cp db.mai.ir db.site1.com

cp db.mai.ir db.site2.com

vi db.site1.com

\%s/mai.ir/site1.com/g

vi db.site2.com

\%s/mai.ir/site2.com/g

systemctl restart bind9

rndc reload

systemctl restart nginx

%add /image page

vi mysites.conf

add after locatino for on server

location /image{
        root /var/www/anisa;
}

mkdir /var/www/anisa/image

echo "<h1>any</h1>" > /var/www/anisa/image/index.html

nginx -t
 
systemctl restart nginx

%see log gile

cd /var/log/nginx

ls

tailf error.log

tailf access.log

%make own log file

vi /etc/nginx/site-enable/mysite.conf

add after address

access_log /vae/log/nginx/access_image.log;
error_log /var/log/nginx/error_image.log;

nginx -t

slstemctl restart nginx

tailf access_image.log

%redirect site1.ir to site1.com

vi /etc/nginx/site-enable/mysite.conf

add 

server {
        listen 192.168.10:80:
        server_name www.site1.ir;
        location / {
                return http://site1.com;
        }
};

nginx -t

systemctl restart nginx

%if the site1.ir not found in centos

in centos

vi /etc/hosts

add after 127.0.0.1

192.168.56.10 www.site1.ir

\subsection*{Nginx Revese Proxy}

balance request to web-server by protocol
, web acceleration
, security

1)Round Robin (cycle)

2)Least Connection (web server with less req choose)

3)IP Hash (same ip same web server)

%adapt load balancing with same weight (Round Robin)

vi /nginx/site-enable/mysites.conf

delete all line then add

upstream loadbalancer(any name) {
        server 127.0.0.1:8080 weight=1;
        server 127.0.0.1:8090 weight=1;
}

server {
        listen 80;
        location / {
                proxy_pass http://loadbalancer;
        }
}

server {
        listen 8080;
        location / {
                root /var/www/anisa/site1;
        }
}

server {
        listen 8090;
        location / {
                root /var/www/anisa/site2;
        }
}

nginx -t 

systemctl restart nginx

ss -ntulp | grep :80

culr http://192.168.56.10

%you can change weight in mysite.conf and see resault

%make this reverce proxy with ip hash

in file mysites.conf in upstream before server 127
add

ip_hash;

nginx -t

systemctl restart nginx

curl

\subsection*{Nginx atuntication}

in /var/www/anisa

mkdir secret

echo "<h1></h1>" > ./secret/index.html

vi /etc/nginx/sites-enable/mysites.conf

delete all line
add

server {
        listen 80;
        location / {
                root /var/www/anisa/site1;
        }
        location /secret {
                root /var/www/anisa;
                allow 192.168.56.20;
                deny all;
                auth_basic "PLZ ENTERe username pas";
                auth_basic_user_file /etc/nginx/farhadpass;
        }
}

nginx -t

htpasswd -c /etc/nginx/farhadpass farhad(username)

cat /etc/nginx/fargadpass

systemctl restart nginx

cd /var/www/anisa/secret

dd if=/dev/zero of=./test.txt bs=500M count=1

in browser http://192.168.56.10/secret/test.txt

in mysites.conf

after auth_basic_user_file add

limit_rate 1m;

nginx -t

systemctl restart nginx


\subsection*{webmin}

%work graphical by linux

cd /etc/apt

vi sources.list

cd sources.list.d

vi webmin.list

deb http://download.webmin.com/download/repasitory sarge contrib

wgen http://www.webmin.com/jcameron.asc

apt-key add jcameron-key.asc

apt-get install webmin

apt-get update

apt-get install webmin

dpkg -l | grep webmin

in sources.list.d

rm -f jcameron-key.asc

in browser https://192.168.56.10:10000


see and manage them


\section*{IPV6}

IPv6 Prefix (subnet mask ipv4)

2001:DB8:3F::/48 --> Network Number (NN) 2 ^ 80 IP

2001:DB8:3F01::/47 --> Single IP becaus
in 81 bit at end of ip (Net id) exist 1 number is all been zero mean is NN

1)-Unicast

2)-Multicast

3)-Anycast

Unicast

Global unicast address --> if first digit of first block == 3 or 2 is Global

Link-local address --> FE80::/64 

Unique local address --> if FC or FD in first

Speecail address --> loopback ::1 (127.0.0.1 in ipv4) or unspecified addr :: (0.0.0.0 in ipv4)

Multicast IPv4 address --> FF at first

transation address

\subsection*{IPv6 IP assignment}

NN = fd00::/64 

in ubuntu

vi /etc/network/interfaces

add

iface enp0s3 inet6 static
        address fd00::10
        netmask 64

systemctl restart networking

in centos

vi /etc/sysconfig/network-scripts/ifcfg-enp0s3

add

IPV6ADDR=fd00::20/64

ping6

rpm -qa | grep bind

yum install -y bind bind-utils

vi /etc/named.conf

in line 13 in oprino

listen-on-v6 ---> any;

allow-query ---> any;

then add new zone

zone "anisa.co.local" {
        type master;
        file "/etc/named/db.anisa.co.local";
};

vi /etc/named/db.anisa.co.local

add

\$TTl 1200

anisa.co.local  IN      SOA     ns.anisa.co.local       root@anisa.co.local (
        10
        1200
        300
        86400
        3600 )
@       IN      NS      ns.anisa.co.local.
ns      IN      AAAA    fd00::20
www     IN      AAAA    fd00::60 
ftp     IN      AAAA    fd00::70 


systemctl restart named

rndc reload

netstat -ntulp | grep :53

dig @localhost AAAA www.anisa.co.local

dig @localhost AAAA ftp.anisa.co.local

NN=fd00::/64

Net ID = fd00:0000:0000:0000 Host Id =0000:0000:0000:0000

define reverse zone:(wiht net id) 

vi /etc/named.conf

zone "0.0.0.0.0.0.0.0.0.0.0.0.0.0.d.f.ip6.arpa" {
        type master;
        file "/etc/named/db.reversev6";
};

vi /etc/named/db.reversev6

\$TTL 1200
\$ORIGIN 0.0.0.0.0.0.0.0.0.0.0.0.0.0.d.f.ip6.arpa.
@       IN      SOA     ns.anisa.co.local. root@anisa.co.local. (
                10
                1200
                300
                86400
                3600 )
@       IN      NS      ns.anisa.co.local. 
0.2.0.0.0.0.0.0.0.0.0.0.0.0.0.0 IN PTR   ns.anisa.co.local.
0.6.0.0.0.0.0.0.0.0.0.0.0.0.0.0 IN PTR   www.anisa.co.local.
0.7.0.0.0.0.0.0.0.0.0.0.0.0.0.0 IN PTR   ftp.anisa.co.local.

systemctl restart named

rndc reload

dig @loclahost -x fd::60

start apache2 (httpd in centos)

set listen to fd::20:80

search [fd::20]

\section*{FTP}

Filte sharing :

samba

ftp --> FTP server :

vsftpd, Pure-FTPd, ProFTPd,other (server side package)

filezilla (client side package)

in ubuntu server

apt-get install vsftpd

in desktop

apt-get install filezilla

yum instal -y epel-release (add epel to package for download)

diffrent between ftp and tftp:

ftp

        authentication
        21
        TCP ==> Reliable, low speed

tfpt

        NO - authentication
        69
        UDP ==>NO - reliable, High speed

diffrent bet ftps and sftp

ftps 

        work with key and tunel

sftp

        encrypt send decrypt


vi /etc/vsftpd.conf

in line 146

userlist_enable=YES
userlist_file=/etc/ftpusers.userlist
userlist_deny=NO

echo maziyar > /etc.ftpusers.userlist (maziyar name of a user)

systemctl restart vsftpd

systemctl status vsftpd

netstat -ntulp | grep :21

to upload file in server you most uncomment permition part

vi /etc/vsftpd.conf

in line 31 write_enable=YES

systemctl restart vsftpd

\section*{fail2ban}

eg. 3 invalid login in 90s ==> 1200s ban

apt-get install fail2ban

cd /etc/fail2ban

vi jail.conf

in line 66 maxretry

systemctl restart fail2ban

\section*{OpenVPN}

yum install epel-release -y

yum install openvpn

apt-get install openvpn

in server(centos)

cd ~

openvpn --genkey --secret tunnel.key

scp tunnel.key root@192.168.56.10:/root/

vi vpnserver.conf

dev tun
ifconfig 10.10.10.20 10.10.10.30
sectet tunnel.key

in ubuntu

cd ~

vi vpnclient.conf

remote 192.168.56.20
dev tun
ifconfig 10.10.10.30 10.10.10.20
sectey tunnel.key

in centos

cd ~

openvpn --config vpnserver.conf

in ubuntu

cd ~

openvpn --config vpnclient.conf

in centos

ping 10.10.10.30

ssh root@10.10.10.30

you can also see wireguard (say it better)

\section*{NFS(Network File System)}

in ubuntu server

apt-get install nfs-kernel-server

mkdir /nfsserver

chmod 777 /nfsserver/

vi /etc/exports

/nfsserver      *(rw)

exportfs -r 

systemctl restart nfs-kernel-server

systemctl status nfs-kernel-server

showmount -e localhost

in centos(client)

mkdir /nfsclient

chmod 777 /nfsclient/

showmount -e 192.168.56.10

vi /etc/fstab

192.168.56.10:/nfdserver        /nfsclient      nfs     default         0       0

in ubuntu

cd /nfsserver

echo "nfs server side" > ./nfsserver.txt

in centos

mount /nfsclient

mount

df

cd /nfsclient

ls

echo "nfs clinet side" > ./nfsclient.txt

in ubuntu

cd /nfsserver

ls

sysstemctl enable nfs-kernel-server

\section*{Squid(caching proxy)}

dpkg -l | grep squid

apt-get install squid3

systemctl restart squid3

systemctl status squid3

netstat -ntulp | grep 3128

(check nat setting)

cat /proc/sys/net/ipv4/ip_forwarding

iptables -t nat -nL

iptables -nL

route -n

cat /etc/resov.conf

ping 8.8.8.8

set http proxy on client with ip 192.168.56.10 with port 3128

vi /etc/squid/squid.conf

in line 1188 (http_access den all)

http_access allow all

systemctl reload squid

vi /etc/squid/squid.conf

in line 3408

cache_dir /usf var/spool/squid 100 16 255 
uncomment thid line

cd /var/spool/squid

vi /etc/squid/squid.conf

in line 988 some acl definded

add

acl GOOGLE dstdomain .google.com

thengo to line 1188

before http_access allow localhost

http_access deny GOOGLE

systemctl reload squid3

vi /etc/squid/squid.conf

inlin 988

acl TUESDAY time T

in line 1186

http_access deny TUESDAY

systemctl reload squid3

vi /etc/squid/squid.conf

in line 988

acl Nerwork1 src 192.168.56.0/24

in line 1168

http_access deny Network1

systemctl reload squid3 

\subsection*{squid athenticaiton}

htpasswd -c /etc/squid/minapass mina

cd /usr/lib/squid

vi /etc/squid/squid.conf

in line 448 before athenticatin tag add

auth_patam basic program /usr/lib/squid/basic_ncsa_auth /etc/squid/minapass

in line 988

acl authenticate proxy_auth REQUIRED

in line 1186

http_access allow authenticate

systemctl reload squid3

\section*{monitoring}

man nc (nc=net cat)

nc -vz localhost 20-30

nc -vz localhost 80-85

nmap localhost

nmap -p 25 localhost

iftop 

nload (switch with up and dowa arrow)

iperf (client server tool)

in server

iperf -s

in client 

iperf -c 192.168.56.30

\section*{Postfix, Procmail, Dovecot}

in ubuntu desktop

apt-get install postfix mailutils


useradd -d /home/mina -m -s /bin/bash mina

useradd -d /home/sina -m -s /bin/bash sina

passwd mina

passwd sina

systemctl status postfix

su - sina

mail

su - mina 

mail

mail sina@ubuntu 
Cc:
Subject: test
hi sina
^d

su - sina

mail
?1
?d
?q

mail root@ubuntu

su - root 

mail

vi /etc/aliases

sina:   root

newaliases

su - mina

mail sina@ubuntu

su - sina

mail

su - root

mail

cat /var/log/mail.log

vi /etc/aliases

del line sina:  root

mail sina@ubuntu

su - sina

mail

vi .forward

mina@ubuntu

su - root

mail sina@ubuntu

su - sina

mail

su -  mina

mail

su - sina

clear .forward

\subsection*{Procmail}(manage mail of user in user home)

in root

apt-get install procmail

two type (mbox=all mail in one file, maildir=all mail in uniq file in one dir)

cd /etc/postfix

vi main.cf

:q

postconf -e "mailbox_command=procmail"

vi main.cf

su - sina

mkdir mail

su - mina

mkdir mail

su - root

mkdir mail

vi /etc/procmailrc

MAILDIR=\$HOME/mail
DEFAULT=\HOME/mail/inbox

mail sina@ubuntu

su - mina

mail sina@ubuntu

su - sina

mail

cd mail

cat inbox

su - root

vi /etc/procmailrc

add / at end of path line line 2

mail mina@ubuntu

su - sina

rm -f mail/inbox

mail mina@ubuntu

su - mina

mail

cd mail

cd inbox

cd new

cat 1551116516.ubuntu

cd ../..

mail -f inbox

?1
?2
?d
?d
?q

\subsection*{Dovecot}

dpkg -l | grep dovecot

apt-get install dovecot-imapd dovecot-pop3d

cd /etc/ssl

we most put 2 file in this path (dovecot-openssl.cnf, mkcert.sh)
but we dont have these

su - root

cd /etc/dovecot

cd conf.d

vi 10-mail.conf

in line 30 mail_location

mail_location = maildir:~/mail/inbox

cd /etc/ssl

vi mkcert.sh

vi dovecot-openssl.cnf

ls certs/ | grep dovecot

ls private/ | grep dovecot

./mkcert.sh

vi /etc/dovecot/conf.d/10-ssl.conf

in line 6 

ssl = required

in line 12,13

ssl_cert = </etc/ssl/certs/dovecot.pem
ssl_key= </etc/ssl/private/dovecot.pem

systemctl restart dovecot

systemctl status dovecot

netstat -ntulp | grep :100

netstat -ntulp | grep :143

su - mina

in Thunderbird (graphical ubuntu desktop)

in account setting

in account action

add mail account

mina
mina@ubuntu
password

su - root

mail mina@ubuntu

su - root

mail mina@ubuntu

graphicaly send a mail to sina

su - sina

mail -f inbox/

?1
?d
?q

\end{document}